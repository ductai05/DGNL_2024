\newpage
\section{Kiến thức}
\label{sec:bigbasic}
\subsection{Phạm vi kiến thức}
\label{sec:phamvikienthuc}
Đôi lời về phạm vi kiến thức kỳ thi ĐGNL 2024
\begin{itemize}
    \item Các bạn nên xem lại \textbf{\hyperref[cautrucbaithi]{cấu trúc bài thi ĐGNL}}.
    \item \textbf{Phân bố điểm} trong bài thi: Tiếng Việt (20 câu), Tiếng Anh (20 câu), Toán học, logic, phân tích số liệu (30 câu), Khoa học tự nhiên (Vật lí, hóa học, sinh học mỗi phần 10 câu), Khoa học xã hội (Địa lí, lịch sử mỗi phần 10 câu)
    \item Có thể thấy: \textbf{Toán học} và \textbf{Khoa học tự nhiên} chiếm \textbf{trọng số cao} hơn trong bài thi so với \textbf{Khoa học xã hội} (lí do là phân môn GDCD không được đưa vào bài thi ĐGNL). Vì vậy lời khuyên cho các bạn học khối xã hội là \textbf{nên cải thiện về mảng Toán và Khoa học tự nhiên} (rất quan trọng nhé).
    \item Về phần \textbf{ngôn ngữ} (tiếng Việt, tiếng Anh): Phạm vi câu hỏi của tiếng Việt rất rộng. Các bạn nên đọc thêm về \textbf{ca dao, tục ngữ Việt Nam} và các \textbf{lỗi chính tả} thường gặp. Với tiếng Anh, các bạn có trình độ \textbf{B1 - B2} (IELTS 4.5 - 5.5) sẽ là khá ổn để làm bài thi.
    \item Về phần \textbf{Khoa học xã hội}, các bạn nên \textbf{chú ý nghe giảng trên lớp} \footnote{Đi du lịch Taiwan bằng tàu điện cao tốc 1000km/h} để tiết kiệm thời gian học. Độ khó của phần Khoa học xã hội sẽ ngang với mức học trên lớp.
    \item Học lệch sẽ kéo điểm bài thi xuống rất nhiều. Nên thay đổi hướng học để phù hợp với hướng xét tuyển đại học của các bạn (nếu \textbf{thi ĐGNL tuyệt đối không học lệch}).
    \item Đối với thi ĐGNL \textbf{Đợt 1}, trong đề sẽ có các câu hỏi vượt qua tiến độ kiến thức các bạn đang học trên lớp. Vì thế, \textbf{học vượt} tiến độ chương trình trên lớp sẽ là một \textbf{lợi thế}. 
    \item Với \textbf{Đợt 2}, \textbf{độ khó của đề sẽ tăng lên} để cân bằng với các bạn chỉ thi Đợt 1 (các bạn thi Đợt 2 được ôn nhiều hơn). Vậy nên thật khó nếu muốn so sánh tham gia đợt thi nào sẽ có lợi hơn. Tuy nhiên, mình \textbf{khuyến khích các bạn thi Đợt 1} (vì Đợt 2 khá sát với kỳ thi tốt nghiệp THPT)
     
\end{itemize}
\subsubsection{Tiếng Việt}
\subsubsection{Tiếng Anh}
\subsubsection{Toán, tư duy logic, phân tích số liệu}
\begin{enumerate}
    \item \textbf{Toán học}
    \begin{itemize}
        \item
        \item Tài liệu tham khảo: \href{https://www.nbv.edu.vn/2023/07/40-chuyen-de-2024.html}{https://www.nbv.edu.vn/2023/07/40-chuyen-de-2024.html}
    \end{itemize}
    
    \item \textbf{Tư duy logic}
    \begin{itemize}
        \item
        \item Tài liệu tham khảo: \href{https://www.nbv.edu.vn/2023/07/40-chuyen-de-2024.html}{https://www.nbv.edu.vn/2023/07/40-chuyen-de-2024.html}
    \end{itemize}
    \item \textbf{Phân tích số liệu}
    \begin{itemize}
        \item
        \item Tài liệu tham khảo: \href{https://www.nbv.edu.vn/2023/07/40-chuyen-de-2024.html}{https://www.nbv.edu.vn/2023/07/40-chuyen-de-2024.html}
    \end{itemize}
\end{enumerate}
\subsubsection{Khoa học tự nhiên}
\begin{enumerate}
    \item \textbf{Vật lý}
    \begin{itemize}
        \item Động học chất điểm, Động lực học chất điểm, Các định luật bảo toàn, Nhiệt động lực học, Chất rắn - lỏng - khí.
        \item Điện tích, \textbf{Điện - Từ trường}, Dòng điện, \textbf{Cảm ứng điện từ}, Khúc xạ ánh sáng, Mắt và dụng cụ quang học.
        \item \textbf{Dao động cơ}, \textbf{Sóng cơ và sóng âm}, \textbf{Dòng điện xoay chiều}, \textbf{Dao động và sóng điện từ}, \textbf{Sóng ánh sáng}, \textbf{Lượng tử ánh sáng}, \textbf{Hạt nhân nguyên tử}.
        \item Tài liệu tham khảo: 
    \end{itemize}
    \item \textbf{Hóa học}
    \begin{itemize}
        \item Nguyên tử, Bảng tuần hoàn, \textbf{Liên kết hóa học}, \textbf{Phản ứng Oxi hóa - khử}, Halogen, Oxi - Lưu huỳnh, \textbf{Tốc độ phản ứng và cân bằng hóa học}. 
        \item \textbf{Sự điện li}, Nito - Photpho, Cacbon - Silic, Hidrocacbon no - không no - thơm, \textbf{Dẫn xuất halogen - ancol - phenol}, Andehit - xeton - axit cacboxylic.
        \item \textbf{Este - Lipit}, \textbf{Cacbohidrat}, \textbf{Amin - Amino Axit - Protein}, \textbf{Polime}, \textbf{Kim loại - Kim loại kiềm, kiềm thổ, nhôm, sắt,...} \textbf{Hóa ứng dụng}.
        \item Tài liệu tham khảo: 
    \end{itemize}
    \item \textbf{Sinh học}
    \begin{itemize}
        \item \textbf{Thế giới sống}, \textbf{Sinh học tế bào}, sinh học vi sinh vật
        \begin{itemize}
            \item Các cấp tổ chức của thế giới sống, Các giới sinh vật.
            \item Tế bào nhân sơ, nhân thực.
            \item \textbf{Nguyên phân}, \textbf{Giảm phân}.
        \end{itemize}
        \item \textbf{Sinh học cơ thể}
        \begin{itemize}
            \item \textbf{Chuyển hóa vật chất và năng lượng} ở thực vật và động vật.
            \item Cảm ứng ở thực vật và thực vật.
            \item \textbf{Sinh sản} ở thực vật và động vật.
        \end{itemize}
        \item \textbf{Di truyền học}, \textbf{Tiến hóa}, \textbf{Sinh thái học}
        \begin{itemize}
            \item Gen, nhân đôi ADN, \textbf{Phiên mã - dịch mã}, \textbf{Đột biến gen}, \textbf{NST và đột biến NST}.
            \item \textbf{Quy luật phân li - phân li độc lập}, \textbf{Tương tác gen}, \textbf{Liên kết - hoán vị gen}, Di truyền liên kết với giới tính, Di truyền ngoài nhân.
            \item Ứng dụng di truyền học: \textbf{biến dị tổ hợp}, đột biến, \textbf{công nghệ tế bào}, công nghệ gen.
            \item Di truyền học người: \textbf{bệnh di truyền phân tử}, đột biến NST
            \item Học thuyết tiến hóa \textbf{Lamarck}, \textbf{Darwin}; \textbf{Học thuyết tiến hóa tổng hợp hiện đại}.
            \item Loài, \textbf{quá trình hình thành loài}, tiến hóa lớn.
            \item Cá thể, \textbf{quần thể}, \textbf{quần xã}, hệ sinh thái, sinh quyển.
        \end{itemize}
        \item Tài liệu tham khảo: 
    \end{itemize}
\end{enumerate}

\subsubsection{Khoa học xã hội}
\begin{enumerate}
    \item\textbf{Địa lí}
    \begin{itemize}
        \item \textbf{Địa lí Việt Nam}: địa lí tự nhiên, dân cư, kinh tế, các vùng kinh tế.
        \item \textbf{Địa lí thế giới}: Mỹ, Nga, EU, Nhật Bản, Trung Quốc, Đông Nam Á.
        \item Tài liệu tham khảo: SGK Địa lí 11, 12 (chương trình cũ)
    \end{itemize}
    \item\textbf{Lịch sử}
    \begin{itemize}
        \item \textbf{Lịch sử Việt Nam} từ năm \textbf{1858} đến năm \textbf{2000}
        \item \textbf{Lịch sử Thế giới} từ 1914 đến 2000
        \item Cách mạng Khoa học - Công nghệ và xu thế Toàn cầu hóa
        \item Tài liệu tham khảo: SGK Lịch sử 11, 12 (chương trình cũ)
    \end{itemize}
\end{enumerate}


\subsection{Kĩ năng làm bài}
\begin{enumerate}
    \item Đề thi chỉ có 120 phút mà lên đến 150 câu (đề thi sẽ khoảng 7 - 9 trang giấy), vì vậy \textbf{kĩ năng đọc hiểu, tìm kiếm thông tin} nên được chú ý cải thiện (đây là một yếu tố quan trọng vì có rất nhiều bạn làm đề không kịp thời gian)
    \item Hãy biết cách \textbf{phân bổ thời gian} hợp lí. Thời gian là vàng.
    \item Câu hỏi \textbf{không sắp xếp theo thứ tự từ dễ đến khó}. Nếu nhìn vào câu hỏi mà bạn chưa biết hướng đi hoặc chưa chắc chắn, hãy \textbf{chừa lại} - để lúc sau làm.
    \item Đừng dành hết thời gian để làm bài. Nên dành ra ít nhất \textbf{10 phút cuối giờ kiểm tra lại đáp án}.
    \item Hãy \textbf{tô trắc nghiệm} một cách chuẩn xác. Vì lượng câu hỏi lớn (150 câu) nên \hyperref[sec:giaidephongthi]{tô trắc nghiệm} cũng cần luyện tập đấy. Nên tô trắc nghiệm khi làm được khoảng \textbf{50 câu}. Trong lúc tô trắc nghiệm, tranh thủ \textbf{kiểm tra lại đáp án}.
\end{enumerate}

\label{sec:kinanglambai}