\section{Kỳ thi Đánh giá năng lực ĐHQG-TPHCM}
\subsection{Tổng quan kỳ thi}
\subsubsection{Mục tiêu}
\begin{itemize}
    \item \textbf{Bài thi ĐGNL ĐHQG-HCM} chú trọng đánh giá các năng lực cơ bản để học đại học của thí sinh: sử dụng ngôn ngữ, tư duy logic, xử lý số liệu, giải quyết vấn đề. Về hình thức, bài thi gồm \textbf{120 câu hỏi trắc nghiệm} khách quan đa lựa chọn với \textbf{thời gian làm bài 150 phút}.
    \item \textbf{Về nội dung}, đề thi cung cấp số liệu, dữ kiện và các công thức cơ bản nhằm đánh giá khả năng suy luận và giải quyết vấn đề, \textbf{không đánh giá khả năng học thuộc lòng}. 
    \item Đề thi được xây dựng cùng cách tiếp cận như đề thi SAT (Scholastic Assessment Test) của Hoa kỳ và đề thi TSA (Thinking Skills Assessment) của Anh
\end{itemize}

\subsubsection{Danh sách các đơn vị xét tuyển bằng kết quả Kỳ thi ĐGNL 2024}
\begin{itemize}
    \item Các \textbf{đơn vị xét tuyển} bằng kết quả kì thi ĐGNL 2024 gồm các trường \textbf{trong ĐHQG-HCM} và các đại học và trường đại học \textbf{ngoài ĐHQG-HCM} (UDN, UEH, NEU, UAH, ...). \cite{dvxettuyen}
    \item Ngoài ra thí sinh còn có thể sử dụng điểm ĐGNL của ĐHQG-HCM (APT) để \textbf{quy đổi} ra điểm ĐGNL của ĐHQG-HN (HSA) theo công thức: \texttt{HSA = 0,1103 x APT}. \cite{apt_to_hsa} 
\end{itemize}
 
\subsubsection{Các mốc thời gian, địa điểm thi} 
\label{sec:thoigiandiadiem}
Các bạn học sinh \textbf{có thể đăng kí thi cả hai đợt ĐGNL ở Lâm Đồng} \cite{thoigian_diadiem} (so với khóa trước thì Đợt 2 không tổ chức ở Lâm Đồng mà phải xuống TP.HCM để thi)
\begin{enumerate}
    \item \textbf{Đợt 1: Ngày 07/04/2024}
    \begin{itemize}
        \item 22/01/2024: Mở đăng ký dự thi ĐGNL đợt 1
        \item 04/3/2024: Kết thúc đăng ký dự thi ĐGNL đợt 1
        \item \textbf{07/4/2024: Tổ chức thi ĐGNL đợt 1 tại 24 tỉnh/thành phố gồm:}
        \begin{itemize}
            \item Trung và Nam Trung Bộ: Thừa Thiên Huế, Đà Nẵng, Quảng Nam, Quảng Ngãi, Bình Định, Phú Yên, Khánh Hòa, Bình Thuận
            \item Tây Nguyên: Đắk Lắk, \textbf{Lâm Đồng - Trường ĐH Đà Lạt}
            \item Đông Nam Bộ: Thành phố Hồ Chí Minh, Bình Dương, Đồng Nai, Bà Rịa – Vũng Tàu, Bình Phước và Tây Ninh
            \item Tây Nam Bộ: Tiền Giang, Bến Tre, Đồng Tháp, Vĩnh Long, An Giang, Cần Thơ, Kiên Giang, Bạc Liêu
        \end{itemize}
        \item 15/4/2024: Thông báo kết quả thi ĐGNL đợt 1.
    \end{itemize}
    \item \textbf{Đợt 2: Ngày 02/6/2024}
    \begin{itemize}
        \item 16/4/2024: Mở đăng ký dự thi ĐGNL đợt 2
        \item 07/5/2024: Kết thúc đăng ký dự thi ĐGNL đợt 2
        \item \textbf{02/6/2024: Tổ chức thi ĐGNL đợt 2 tại 12 tỉnh/thành phố gồm:}
        \begin{itemize}
            \item Trung và Nam Trung Bộ: Thừa Thiên Huế, Đà Nẵng, Bình Định, Khánh Hòa
            \item Tây Nguyên: Đắk Lắk, \textbf{Lâm Đồng}
            \item Đông Nam Bộ: Thành phố Hồ Chí Minh, Bà Rịa- Vũng Tàu, Đồng Nai, Bình Dương
            \item Tây Nam Bộ: Tiền Giang, An Giang
        \end{itemize}
        \item 10/6/2024: Thông báo kết quả thi ĐGNL đợt 2
    \end{itemize}
\end{enumerate}

\subsection{Các trang web liên quan đến kì thi}
\begin{itemize}
    \item \textbf{Cổng thông tin đăng kí:} \href{https://thinangluc.vnuhcm.edu.vn/dgnl/}{https://thinangluc.vnuhcm.edu.vn/dgnl/} \cite{webdangki}
    \item Các bạn nên tham khảo các thông tin trên website của Trung tâm khảo thí và Đánh giá chất lượng đào tạo ĐHQG-HCM: \href{http://cete.vnuhcm.edu.vn/thi-danh-gia-nang-luc.html}{http://cete.vnuhcm.edu.vn/thi-danh-gia-nang-luc.html} \cite{cetevnuhcm}
\end{itemize}

\subsection{Cấu trúc bài thi ĐGNL}
\label{cautrucbaithi}
Cấu trúc của bài thi ĐGNL gồm \textbf{3 phần}: Sử dụng ngôn ngữ; Toán học, tư duy logic và
phân tích số liệu; và Giải quyết vấn đề.
\subsubsection{Phần 1. Sử dụng ngôn ngữ (40 câu)}
\paragraph{a) Tiếng Việt (20 câu):}Đánh giá năng lực đọc hiểu văn bản và sử dụng tiếng Việt, và khả năng cảm thụ, phân tích các tác phẩm văn học. Đề thi tích hợp nhiều kiến thức về ngữ văn, đòi hỏi thí sinh nắm vững những kỹ năng thực hành tiếng Việt để áp dụng vào giải quyết các vấn đề liên quan.
\par

\begin{table}[H]
\begin{tabular}{|l|l|}
\hline
\textbf{Nội dung đánh giá} & \textbf{Mô tả} \\ \hline
Hiểu biết văn học & \begin{tabular}[c]{@{}l@{}}Đánh giá khả năng hiểu các kiến thức văn học cơ bản như: phong cách\\ sáng tác của các tác giả tiêu biểu, nội dung và hình thức nghệ thuật của \\ tác phẩm; vai trò của tác giả, tác phẩm đối với lịch sử văn học.\end{tabular} \\ \hline
Sử dụng tiếng Việt & \begin{tabular}[c]{@{}l@{}}Đánh giá khả năng nhận biết vấn đề về sử dụng tiếng Việt như: xác định \\ những từ viết không đúng quy tắc chính tả, những từ sử dụng sai, những \\ câu mắc lỗi ngữ pháp diễn đạt; nhận biết cấu tạo từ, các biện pháp tu từ, \\ các vấn đề thuộc về ngữ pháp câu, các thành phần trong câu, phép liên \\ kết câu,…\end{tabular} \\ \hline
Đọc hiểu văn bản & \begin{tabular}[c]{@{}l@{}}Đánh giá khả năng phân loại đặc trưng phong cách (phong cách thể loại, \\ phong cách tác giả, phong cách chức năng ngôn ngữ, …), xác định ý \\ nghĩa của từ/câu trong văn bản, cách tổ chức văn bản, các thủ pháp nghệ \\ thuật được sử dụng, nội dung và tư tưởng của văn bản.\end{tabular} \\ \hline
\end{tabular}
\end{table}

\paragraph{b) Tiếng Anh (20 câu):}Đánh giá năng lực sử dụng tiếng Anh tổng quát ở cấp độ A2-B1 theo khung năng lực ngoại ngữ 6 bậc, thông qua các nội dung: lựa chọn cấu trúc câu, nhận diện lỗi sai, đọc hiểu câu, đọc hiểu đoạn văn:
\par

\begin{table}[H]
\begin{tabular}{|l|l|}
\hline
\textbf{Nội dung đánh giá} & \textbf{Mô tả} \\ \hline
Lựa chọn cấu trúc câu & \begin{tabular}[c]{@{}l@{}}Đánh giá khả năng hiểu và áp dụng các cấu trúc câu thông qua việc \\ yêu cầu thí sinh chọn từ/cụm từ có cấu trúc phù hợp để điền vào \\ khoảng trống.\end{tabular} \\ \hline
Nhận diện lỗi sai & \begin{tabular}[c]{@{}l@{}}Đánh giá khả năng hiểu các kiến thức ngữ pháp và áp dụng để giải\\ quyết vấn đề thông qua việc nhận diện lỗi sai trong những phần \\ được gạch chân.\end{tabular} \\ \hline
Đọc hiểu câu & \begin{tabular}[c]{@{}l@{}}Đánh giá khả năng đọc hiểu câu và khả năng áp dụng kiến thức ngữ\\ pháp đã học thông qua việc chọn câu có nghĩa gần nhất với \\ câu đã cho.\end{tabular} \\ \hline
Đọc hiểu đoạn văn & \begin{tabular}[c]{@{}l@{}}Đánh giá khả năng hiểu và áp dụng kiến thức ngữ pháp cũng như \\ kỹ năng đọc lướt để lấy thông tin (skimming) và đọc kỹ để tìm chi \\ tiết (scanning), cụ thể: đọc lướt để trả lời câu hỏi lấy ý chính (main \\ idea), đọc kỹ để trả lời các câu hỏi tham chiếu (reference), câu hỏi \\ chi tiết (detail), câu hỏi từ vựng (vocabulary), câu hỏi suy luận \\ (inference).\end{tabular} \\ \hline
\end{tabular}
\end{table}

\subsubsection{Phần 2. Toán học, tư duy logic và phân tích số liệu (30 câu)}
\paragraph{}Đánh giá khả năng áp dụng các kiến thức toán học; khả năng tư duy logic; khả năng diễn giải, so sánh phân tích số liệu:
\par

\begin{table}[H]
\begin{tabular}{|l|l|}
\hline
\textbf{Nội dung đánh giá} & \textbf{Mô tả} \\ \hline
Toán học & \begin{tabular}[c]{@{}l@{}}Đánh giá khả năng hiểu và áp dụng các kiến thức toán học trong chương \\ trình giáo khoa trung học phổ thông thuộc các nội dung: ứng dụng của \\ đạo hàm để khảo sát hàm số, số phức (tìm phần thực, phần ảo Mô-đun, \\ không có phương trình bậc 2, không có dạng lượng giác), hình học thuần \\ túy, hình học tọa độ, tích phân và ứng dụng của tích phân, tổ hợp và xác \\ suất, hàm số logarit, giải toán bằng cách lập hệ phương trình, giải hệ \\ phương trình tuyến tính suy biến.\end{tabular} \\ \hline
Tư duy logic & \begin{tabular}[c]{@{}l@{}}Đánh giá khả năng suy luận logic thông qua các hình thức logic đơn lẻ \\ và nhóm logic tình huống. Dựa vào các thông tin được cung cấp trong \\ mỗi tình huống logic cùng với kỹ năng suy luận và phân tích, thí sinh \\ tìm phương án khả thi cho các giả định được đưa ra.\end{tabular} \\ \hline
Phân tích số liệu & \begin{tabular}[c]{@{}l@{}}Đánh giá khả năng đọc và phân tích số liệu thực tế thông qua các sơ đồ \\ và các bảng số liệu. Các sơ đồ và bảng biểu xuất hiện trong đề thi gồm: \\ biểu đồ tròn, biểu đồ Venn, biểu đồ cột, biểu đồ đường, biểu đồ dạng \\ bảng số liệu.\end{tabular} \\ \hline
\end{tabular}
\end{table}

\subsubsection{Phần 3. Giải quyết vấn đề (50 câu)}
\paragraph{}Đánh giá khả năng hiểu các kiến thức giáo khoa cơ bản và áp dụng để giải quyết các vấn đề cụ thể thuộc năm lĩnh vực, gồm ba lĩnh vực khoa học tự nhiên (hóa học, vật lý, sinh học) và hai lĩnh vực khoa học xã hội (địa lí, lịch sử):
\par

\begin{table}[H]
\begin{tabular}{|l|l|}
\hline
\textbf{Nội dung đánh giá} & \textbf{Mô tả} \\ \hline
\begin{tabular}[c]{@{}l@{}}Lĩnh vực khoa học tự nhiên \\ (hóa học, vật lí, sinh học)\end{tabular} & \begin{tabular}[c]{@{}l@{}}Các câu hỏi đơn lẻ đánh giá khả năng hiểu các kiến thức giáo khoa \\ cơ bản liên quan đến ba lĩnh vực khoa học tự nhiên: hóa học, vật lý, \\ sinh học.\\ Các nhóm câu hỏi tình huống đánh giá khả năng đọc, tư duy, suy \\ luận logic về hóa học, vật lí, sinh học thông qua dữ kiện được cung \\ cấp trong các bài đọc và kiến thức đã học; đánh giá khả năng áp \\ dụng các kiến thức phổ thông để giải quyết các vấn đề liên quan.\end{tabular} \\ \hline
\begin{tabular}[c]{@{}l@{}}Lĩnh vực khoa học xã hội \\ (địa lí, lịch sử)\end{tabular} & \begin{tabular}[c]{@{}l@{}}Các câu hỏi đơn lẻ đánh giá khả năng  hiểu kiến thức giáo khoa cơ \\ bản liên quan đến lĩnh vực khoa học xã hội: địa lý, lịch sử.\\ Các nhóm câu hỏi tình huống đánh giá khả năng đọc, tư duy, suy \\ luận logic về địa lý, lịch sử thông qua dữ kiện được cung cấp \\ trong các bài đọc, kiến thức đã học hoặc kiến thức thực tế; \\ năng lực áp dụng các kiến thức phổ thông để giải quyết các \\ vấn đề liên quan.\end{tabular} \\ \hline
\end{tabular}
\end{table}


